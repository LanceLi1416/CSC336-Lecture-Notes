\chapter{Introduction}

\section{Motivation}

\[
    \text{Math Textbook} + \text{Laptop} + \text{Coding} \overset{?}{\implies} \text{Compute Accurate Solution}
\]

Consider the McLaurin series expansion of the function \( f(x) = e^x \): \begin{align*}
    e^x & = 1 + x + \frac{x^2}{2!} + \frac{x^3}{3!} + \cdots \\
        & = \sum_{n=0}^{\infty} \frac{x^n}{n!}
\end{align*}

The issue is that we cannot compute to infinity. We need to introduce partial sums \[
    S_n = \sum_{i=0}^{n} \frac{x^i}{i!}
\]

We could iterate over \( n \) until \( \left| S_n - S_{n-1} \right| < \text{tolerance} \).

\begin{table}[ht!]
    \centering
    \begin{tabular}{c|ccccc}
        \( x \)                   & 0 & 1  & 10 & 20 & 40  \\
        Num Terms to ``Converge'' & 2 & 13 & 42 & 69 & 104 \\
    \end{tabular}
\end{table}

We observe that the running time is dependent on the value of \( x \). We need to find a better way to compute the sum -- with more consistent running time.

    {~~~}

Using the python program,
\begin{itemize}
    \item When \( x = -30 \), convergence happened after 97 terms, to \( -6.0 \times 10^{-5} \).
    \item When \( x = -40 \), convergence happened after 124 terms, to approximately \( -5.9 \times 10^0 \).
\end{itemize}

\begin{figure}[ht!]
    \centering
    \begin{tikzpicture}
        \draw[->] (-5,0) -- (1,0) node[right] {\( x \)};
        \draw[->] (0,-2) -- (0,2) node[above] {\( y \)};

        % y = e^x
        \draw[domain=-5:1,smooth,variable=\x,red] plot ({\x},{exp(\x)}) node[right] {\( e^x \)};

        % Dot at x = -30
        \draw[blue,fill=blue] (-3,0) circle (0.05) node[below] {\( -30 \)};
        \draw[blue,fill=blue] (-4,0) circle (0.05) node[below] {\( -40 \)};
    \end{tikzpicture}
\end{figure}

Clearly, we have inaccuracy when \( x = -40 \), as \( 0 < e^x < 1 \) for all \( x < 0 \). The math textbooks'
s techniques does not always provide good computational algorithms.

    {~~~}

\textbf{Course goal:} Show computational algorithms and discuss why they are good.

\begin{example}[\( e^x \) Better Algorithm]
    A better algorithm is as follows

    \begin{itemize}
        \item Find \( k \) such that \( r = \frac{x}{k} \) exactly with \( \|r \| < 1 \).
        \item Compute $e^{r} = e^{x/k}$ using the McLaurin series.
        \item Then, $e^x = (e^r)^k$.
    \end{itemize}
\end{example}

\begin{remark}[Error due to Catasrophic Cancellation]
    When we subtract two numbers that are very close to each other, we lose precision.
\end{remark}

\section{Topics}

\begin{itemize}
    \item Computer Arithmetic and Computational Errors (Chap. 1)
          \begin{itemize}
              \item Floating Point Arithmetic
              \item Two Concepts
                    \begin{itemize}
                        \item The conditioning of a math problem
                        \item the numerical stability of an algorithm
                    \end{itemize}
          \end{itemize}

    \item Solving Systems of Linear Equations (Chap. 2)

          \begin{itemize}
              \item Solve \( A x = b \) for \( x \)
          \end{itemize}

    \item Solving Non-linear Equations (Chap. 5)

          Fine \( x \) s.t. \( f(x) = 0 \) or \( g(x) = 0 \) or \( f(x) = g(x) \).

    \item Interpolation (Chap. 7)
          \begin{itemize}
              \item  Given the set of data \[
                        \left\{ (t_i, y_i) \right\}_{i=0}^{n}
                        \qquad \text{ or } \qquad
                        \left\{ (t_i, f(t_i)) \right\}_{i=0}^{n}
                    \] come up with a function $g(t)$ that approximates the data.
          \end{itemize}
\end{itemize}