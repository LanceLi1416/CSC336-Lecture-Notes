\usepackage{algorithm}
\usepackage{algorithmicx}
\usepackage{algpseudocode}
\usepackage{amsfonts}
\usepackage{amsmath}
\usepackage{amsthm}
\usepackage{amssymb}
\usepackage{booktabs}
\usepackage{etoolbox}
\usepackage{float}
\usepackage{graphicx}
\usepackage{hyperref}
\usepackage{lipsum}
\usepackage{longtable}
\usepackage{minted}
\usepackage{pgffor}
\usepackage{subcaption}
\usepackage{tabularx}
\usepackage{tikz, pgfplots}
\usepackage{url}
\usepackage{xfrac}

\usepackage[english]{babel}
\usepackage[T1]{fontenc}
\usepackage[
    letterpaper,
    inner=1.25in,
    outer=1in,
    top=1in,
    bottom=1in
]{geometry}
\usepackage[scaled]{helvet}
\usepackage[utf8]{inputenc}
\usepackage[most]{tcolorbox}
% \usepackage[dvipsnames]{xcolor}

%%%%%%%%%%%%%%%%%%%%%%%%%%%%%%%%%%%%%%%%%%%%%%%%%%%%%%%%%%%%%%%%%%%%%%%%%%%%%%%%
% Colour
%%%%%%%%%%%%%%%%%%%%%%%%%%%%%%%%%%%%%%%%%%%%%%%%%%%%%%%%%%%%%%%%%%%%%%%%%%%%%%%%
\input{config/colors.tex}

% Light-Blue-700
\definecolor{primary}{HTML}{0288D1}

% Hyperref ---------------------------------------------------------------------
\hypersetup{
    colorlinks,
    linkcolor=black,
    filecolor=Pink-300,
    urlcolor=Blue-700,
    citecolor=black,
}

%%%%%%%%%%%%%%%%%%%%%%%%%%%%%%%%%%%%%%%%%%%%%%%%%%%%%%%%%%%%%%%%%%%%%%%%%%%%%%%%
% Custom commands
%%%%%%%%%%%%%%%%%%%%%%%%%%%%%%%%%%%%%%%%%%%%%%%%%%%%%%%%%%%%%%%%%%%%%%%%%%%%%%%%
\newcommand{\by}[1]{\hfill--- #1}

\makeatletter
\renewcommand*\env@matrix[1][*\c@MaxMatrixCols c]{%
    \hskip -\arraycolsep
    \let\@ifnextchar\new@ifnextchar
    \array{#1}}
\makeatother

% TODO: Add more custom commands here
\newcommand{\R}{\mathbb{R}}
\renewcommand{\P}{\mathbb{P}}

\renewcommand{\vec}[1]{\underline{#1}}

\pgfplotsset{compat=1.18}

%%%%%%%%%%%%%%%%%%%%%%%%%%%%%%%%%%%%%%%%%%%%%%%%%%%%%%%%%%%%%%%%%%%%%%%%%%%%%%%%
% Text styles
%%%%%%%%%%%%%%%%%%%%%%%%%%%%%%%%%%%%%%%%%%%%%%%%%%%%%%%%%%%%%%%%%%%%%%%%%%%%%%%%
% Font -------------------------------------------------------------------------
% \renewcommand{\familydefault}{\sfdefault} % Sans-serif font
\renewcommand{\familydefault}{\rmdefault} % Serif font

% Text styles ------------------------------------------------------------------
\DeclareTextFontCommand{\term}{\color{Orange-700}\bfseries}

\DeclareTextFontCommand{\bold}{\bfseries}
\DeclareTextFontCommand{\italic}{\itshape}

\renewcommand{\textbf}[1]{\textcolor{Red-800}{\bfseries #1}}
\renewcommand{\textit}[1]{\textcolor{Light-Blue-400}{\itshape #1}}

% Paragraph spacing ------------------------------------------------------------
% \setlength{\parindent}{0pt} % No paragraph indentation
% \setlength{\parskip}{1em} % Paragraph spacing

%%%%%%%%%%%%%%%%%%%%%%%%%%%%%%%%%%%%%%%%%%%%%%%%%%%%%%%%%%%%%%%%%%%%%%%%%%%%%%%%
% Environments
%%%%%%%%%%%%%%%%%%%%%%%%%%%%%%%%%%%%%%%%%%%%%%%%%%%%%%%%%%%%%%%%%%%%%%%%%%%%%%%%
\renewenvironment{quote}
{\list{}{\rightmargin=0.5cm\leftmargin=0.5cm}\item\relax\itshape}
{\endlist}

% Lists ------------------------------------------------------------------------
% itemize
\renewcommand{\labelitemi}{\textcolor{primary}{\textbullet}}
\renewcommand{\labelitemii}{\textcolor{primary}{\textbullet}}
\renewcommand{\labelitemiii}{\textcolor{primary}{\textbullet}}
\renewcommand{\labelitemiv}{\textcolor{primary}{\textbullet}}

% enumerate
\newcommand{\cnum}[2]{
    \tikz[baseline=(c.base)]
    \node[circle,fill=#2,minimum size=0.5cm,inner sep=1pt](c)
    {\color{white}\bfseries\fontsize{8}{8}#1};}

\renewcommand{\labelenumi}{\cnum{\arabic{enumi}}{primary}}
\renewcommand{\labelenumii}{\cnum{\alph{enumii}}{primary}}
\renewcommand{\labelenumiii}{\cnum{\roman{enumiii}}{primary}}
\renewcommand{\labelenumiv}{\cnum{\Alph{enumiv}}{primary}}

% amsthm -----------------------------------------------------------------------
\newcommand\fancybox[3]{%
    \tcbset{mybox/.style={enhanced,boxsep=0mm,opacityfill=0,overlay={
                    \coordinate (X) at ([xshift=-1mm, yshift=-1.5mm]frame.north west);
                    \node[align=right, text=#1, text width=2.5cm, anchor=north east] at (X) {\bf#2};
                    \draw[line width=0.5mm, color=#1] (frame.north west) -- (frame.south west);
                }}} \begin{tcolorbox}[mybox] #3 \end{tcolorbox}}

\tcbuselibrary{theorems,skins,hooks}
\NewDocumentCommand\thmbox{m O{\Large #1} O{Gray-100} O{primary} O{number within=section}}
{
    \newtcbtheorem[#5]{#1}{\large #2}
    {%
        enhanced,
        breakable,
        colback = #3,
        frame hidden,
        boxrule = 0sp,
        borderline west = {2pt}{0pt}{#4},
        sharp corners,
        detach title,
        before upper = \tcbtitle\par\smallskip,
        coltitle = #4,
        fonttitle = \bfseries,
        description font = \mdseries,
        separator sign none,
        segmentation style={solid, #4}
    }
    {th}
}

% Universal counter for theorem environments
\newcounter{universal}

% Corollary --------------------------------------------------------------------
\thmbox{Corollary} [Corollary] [Purple-50] [Purple-800]
[use counter=universal, number within=section]

\newenvironment{corollary}[1][] {\begin{Corollary}{#1}{}} {\end{Corollary}}

% Definition -------------------------------------------------------------------
\thmbox{Definition} [Definition] [Light-Blue-50] [Light-Blue-800]
[use counter=universal, number within=section]

\newenvironment{definition}[1][] {\begin{Definition}{#1}{}} {\end{Definition}}

% Example ----------------------------------------------------------------------
\theoremstyle{definition}
\newtheorem*{Example}{\color{primary}Example}

\newenvironment{example}
{\begin{Example}}
        % {\begin{Example}\setlength{\parindent}{0pt}}
        {\hfill\ensuremath{\color{primary}\diamondsuit}\end{Example}}

% Lemma ------------------------------------------------------------------------
\thmbox{Lemma} [Lemma] [Green-50] [Green-800]
[use counter=universal, number within=section]

\newenvironment{lemma}[1][] {\begin{Lemma}{#1}{}} {\end{Lemma}}

% Remark -----------------------------------------------------------------------
\thmbox{Remark} [Remark] [Gray-200] [Gray-700] [no counter]

\newenvironment{remark}[1][] {\begin{Remark}{#1}{}} {\end{Remark}}

% Proposition ------------------------------------------------------------------
\thmbox{Proposition} [Proposition] [Orange-50] [Orange-800]
[use counter=universal, number within=section]

\newenvironment{proposition}[1][] {\begin{Proposition}{#1}{}} {\end{Proposition}}

% Theorem ----------------------------------------------------------------------
\thmbox{Theorem} [Theorem] [Red-50] [Red-800]
[use counter=universal, number within=section]

\newenvironment{theorem}[1][] {\begin{Theorem}{#1}{}} {\end{Theorem}}

% Note -------------------------------------------------------------------------
\newtcolorbox{note}[1][] {
    colback=Gray-100,
    colframe=Gray-200,
    coltitle=Gray-700,
    title=\bold{Note: #1},
    fonttitle=\bfseries,
    breakable
}

% Algorithms -------------------------------------------------------------------
% \Call
\algrenewcommand\Call[2]{\textproc{\color{primary}\textsc{#1}}(#2)}

% \Else
\algrenewcommand\algorithmicelse {\textsc{\color{primary}Else}}

% \For
\algrenewcommand\algorithmicfor {\textsc{\color{primary}For}}
\algtext*{EndFor}

% \Function
\algrenewcommand\algorithmicfunction {\textsc{\color{primary}Function}}
\algtext*{EndFunction}

% \If
\algrenewcommand\algorithmicif {\textsc{\color{primary}If}}
\algtext*{EndIf}

% \Return
\algrenewcommand\algorithmicreturn {\textsc{\color{primary}Return}}

% \To
\newcommand{\To}{\textsc{\color{primary}To}\xspace}

% \DownTo
\newcommand{\DownTo}{\textsc{\color{primary}DownTo}\xspace}

% \While
\algrenewcommand\algorithmicwhile {\textsc{\color{primary}While}}
\algtext*{EndWhile}

% Proof ------------------------------------------------------------------------
\renewcommand{\qedsymbol}{\ensuremath\blacksquare}

%%%%%%%%%%%%%%%%%%%%%%%%%%%%%%%%%%%%%%%%%%%%%%%%%%%%%%%%%%%%%%%%%%%%%%%%%%%%%%%%
% Heading styles
%%%%%%%%%%%%%%%%%%%%%%%%%%%%%%%%%%%%%%%%%%%%%%%%%%%%%%%%%%%%%%%%%%%%%%%%%%%%%%%%
\usepackage[explicit]{titlesec}

% Part -------------------------------------------------------------------------
\titleformat{\part}[display]
{\Large\bf} {\textsc{\partname~\thepart}} {12pt} {
    \Huge\textsc{#1}
} [\thispagestyle{empty}]

% Chapter ----------------------------------------------------------------------
\newtcolorbox{titlecolorbox}[1]{
    coltext=white,
    colframe=primary,
    colback=primary,
    boxrule=0pt,
    arc=0pt,
    notitle,
    width=4.3em,
    height=2.4ex,
    before=\hfill
}

\makeatletter
\let\old@rule\@rule
\def\@rule[#1]#2#3{\textcolor{primary}{\old@rule[#1]{#2}{#3}}}
\makeatother

\titleformat{\chapter}[display]
{\Huge} {} {0pt} {
    \begin{titlecolorbox}{}
        {\large\MakeUppercase{\bf\chaptername}}
    \end{titlecolorbox}
    \vspace*{-3.19ex}\noindent\rule{\textwidth}{0.4pt}
    \parbox[b]{\dimexpr\textwidth-4.8em\relax}{\raggedright\textsc{#1}}
    {\hfill\fontsize{50}{40}\selectfont{\color{primary}\thechapter}}
} [\thispagestyle{empty}]

\titleformat{name=\chapter,numberless}[display]
{\Huge} {} {0pt} {
    \vspace*{-3.19ex}\noindent\rule{\textwidth}{0.4pt}
    \parbox[b]{\dimexpr\textwidth-4.8em\relax}{\raggedright\MakeUppercase{#1}}
} [\thispagestyle{empty}]

\titlespacing*{\chapter}{0pt}{-20pt}{20pt}

% Section ----------------------------------------------------------------------
\titleformat{\section}[hang]{\Large\bfseries}%
{\rlap{ \color{primary} \rule[-6pt] {\textwidth} {0.4pt} }
    \colorbox{primary} {
        \raisebox{0pt}[13pt][3pt] {
            \makebox[60pt]{ \selectfont\color{white}{\thesection} }
        }
    }
} {15pt} {\color{primary}#1}

% Subsection -------------------------------------------------------------------
\titleformat{\subsection}[hang]{\Large\bfseries}
{\color{primary}{\thesubsection}} {10pt} {\color{primary}#1}

%%%%%%%%%%%%%%%%%%%%%%%%%%%%%%%%%%%%%%%%%%%%%%%%%%%%%%%%%%%%%%%%%%%%%%%%%%%%%%%%
% Header and footer
%%%%%%%%%%%%%%%%%%%%%%%%%%%%%%%%%%%%%%%%%%%%%%%%%%%%%%%%%%%%%%%%%%%%%%%%%%%%%%%%
\usepackage{fancyhdr}

\fancypagestyle{fancy} {
    % Clear header/footer
    \fancyhf{}
    % Page number (left of even, right of odd)
    \fancyhead[LE,RO]{\thepage}
    % Chapter name (right of even)
    \fancyhead[RE]{\nouppercase{\leftmark}}
    % Section name (left of odd)
    \fancyhead[LO]{\nouppercase{\rightmark}}
    % No header rule
    \renewcommand{\headrulewidth}{0pt}
}


% Define new page style for frontmatter
\fancypagestyle{frontmatter} {
    % Clear header/footer
    \fancyhf{}
    % No header rule
    \renewcommand{\headrulewidth}{0pt}
    % Page in footer, centred
    \fancyfoot[C]{\thepage}
}

%%%%%%%%%%%%%%%%%%%%%%%%%%%%%%%%%%%%%%%%%%%%%%%%%%%%%%%%%%%%%%%%%%%%%%%%%%%%%%%%
% Table of contents
%%%%%%%%%%%%%%%%%%%%%%%%%%%%%%%%%%%%%%%%%%%%%%%%%%%%%%%%%%%%%%%%%%%%%%%%%%%%%%%%
\usepackage{blindtext}
\usepackage{framed}
\usepackage{titletoc}

\patchcmd{\tableofcontents}{\contentsname}{\contentsname}{}{}

\newtoggle{isUnnumberedChapter}
\togglefalse{isUnnumberedChapter} % Default state

\renewenvironment{leftbar}
{\def\FrameCommand{\hspace{6em}%
        {\color{primary}\vrule width 2pt depth 6pt}\hspace{1em}}%
    \MakeFramed{\parshape 1 0cm \dimexpr\textwidth-6em\relax\FrameRestore}\vskip2pt%
}
{\endMakeFramed}

\titlecontents{chapter}[0em]
{\vspace*{2\baselineskip}}
{\parbox{4.5em}{%
        \hfill\Huge\bfseries\color{primary}\thecontentslabel}%
    \vspace*{-2.3\baselineskip}\leftbar\textbf{\color{primary}\small\chaptername~\thecontentslabel}\\
}{}{\endleftbar}

\titlecontents{section}[8.4em]
{\contentslabel{3em}}{}{}
{\hspace{0.5em}\nobreak\itshape\color{primary}\contentspage}

\titlecontents{subsection}[11.4em]
{\contentslabel{3em}}{}{}
{\hspace{0.5em}\nobreak\itshape\color{primary}\contentspage}

%%%%%%%%%%%%%%%%%%%%%%%%%%%%%%%%%%%%%%%%%%%%%%%%%%%%%%%%%%%%%%%%%%%%%%%%%%%%%%%%
% Glossary
%%%%%%%%%%%%%%%%%%%%%%%%%%%%%%%%%%%%%%%%%%%%%%%%%%%%%%%%%%%%%%%%%%%%%%%%%%%%%%%%
% TODO: Glossary configuration

%%%%%%%%%%%%%%%%%%%%%%%%%%%%%%%%%%%%%%%%%%%%%%%%%%%%%%%%%%%%%%%%%%%%%%%%%%%%%%%%
% Bibliography
%%%%%%%%%%%%%%%%%%%%%%%%%%%%%%%%%%%%%%%%%%%%%%%%%%%%%%%%%%%%%%%%%%%%%%%%%%%%%%%%
\usepackage{csquotes}

\usepackage[
    style=ieee,
    citestyle=ieee,
    sorting=nyt,
    sortcites=true,
    autopunct=true,
    autolang=hyphen,
    hyperref=true,
    abbreviate=false,
    backref=true,
    backend=biber,
    defernumbers=true
]{biblatex}

\addbibresource{bibliography/template.bib}
\addbibresource{bibliography/reference.bib}
\defbibheading{bibempty}{}

%%%%%%%%%%%%%%%%%%%%%%%%%%%%%%%%%%%%%%%%%%%%%%%%%%%%%%%%%%%%%%%%%%%%%%%%%%%%%%%%
% Index
%%%%%%%%%%%%%%%%%%%%%%%%%%%%%%%%%%%%%%%%%%%%%%%%%%%%%%%%%%%%%%%%%%%%%%%%%%%%%%%%
\usepackage{calc}
\usepackage{makeidx}

\makeindex

%%%%%%%%%%%%%%%%%%%%%%%%%%%%%%%%%%%%%%%%%%%%%%%%%%%%%%%%%%%%%%%%%%%%%%%%%%%%%%%%
% Tikz Externalize
%%%%%%%%%%%%%%%%%%%%%%%%%%%%%%%%%%%%%%%%%%%%%%%%%%%%%%%%%%%%%%%%%%%%%%%%%%%%%%%%
\usetikzlibrary{external}
\tikzexternalize[prefix=tikz/]

% Disable externalization globally, and only enable it for `tikzpicture'
\tikzexternaldisable
\BeforeBeginEnvironment[label]{tikzpicture}{\tikzexternalenable}
\AfterEndEnvironment[label]{tikzpicture}{\tikzexternaldisable}

%%%%%%%%%%%%%%%%%%%%%%%%%%%%%%%%%%%%%%%%%%%%%%%%%%%%%%%%%%%%%%%%%%%%%%%%%%%%%%%%
% Title page
%%%%%%%%%%%%%%%%%%%%%%%%%%%%%%%%%%%%%%%%%%%%%%%%%%%%%%%%%%%%%%%%%%%%%%%%%%%%%%%%
\usetikzlibrary{calc}
\usetikzlibrary{shapes.geometric}
\usepackage{anyfontsize}
\newcommand{\frontpage}[3]{
    \tikzset{external/export next=false}
    \begin{tikzpicture}[remember picture, overlay]
        % Background
        \fill[primary] (current page.south west) rectangle (current page.north east);

        \foreach \i in {2.5,...,22} {
                \node[rounded corners,primary!60,draw,regular polygon,regular polygon sides=6, minimum size=\i cm,ultra thick] at ($(current page.west)+(2.5,-5)$) {} ;
            }

        % Background Polygon
        \foreach \i in {0.5,...,22} {
                \node[rounded corners,primary!60,draw,regular polygon,regular polygon sides=6, minimum size=\i cm,ultra thick] at ($(current page.north west)+(2.5,0)$) {} ;
            }

        \foreach \i in {0.5,...,22} {
                \node[rounded corners,primary!90,draw,regular polygon,regular polygon sides=6, minimum size=\i cm,ultra thick] at ($(current page.north east)+(0,-9.5)$) {} ;
            }

        \foreach \i in {21,...,6} {
                \node[primary!85,rounded corners,draw,regular polygon,regular polygon sides=6, minimum size=\i cm,ultra thick] at ($(current page.south east)+(-0.2,-0.45)$) {} ;
            }

        % Title
        \node[left,primary!5,minimum width=0.625*\paperwidth,minimum height=3cm, rounded corners] at ($(current page.north east)+(0,-9.5)$) {
            {\fontsize{25}{30} \selectfont \bfseries #1}
        };

        % Subtitle
        \node[left,primary!10,minimum width=0.625*\paperwidth,minimum height=2cm, rounded corners] at ($(current page.north east)+(0,-11)$) {
            {\huge \italic{#2}}
        };

        % Author
        \node[left,primary!5,minimum width=0.625*\paperwidth,minimum height=2cm, rounded corners] at ($(current page.north east)+(0,-13)$) {
            {\Large \textsc{#3}}
        };

        % Year
        \node[rounded corners,fill=primary!70,text =primary!5,regular polygon,regular polygon sides=6, minimum size=2.5 cm,inner sep=0,ultra thick] at ($(current page.west)+(2.5,-5)$) {\LARGE \bfseries \the\year{}};
    \end{tikzpicture}
}